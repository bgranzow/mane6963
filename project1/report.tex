\documentclass[11pt]{article}
\usepackage{amssymb}
\usepackage{amsthm}
\usepackage{amsmath}
\usepackage{listings}
\usepackage{color}
\usepackage{graphicx}
\usepackage{gensymb}
\usepackage[margin=1.0in]{geometry}

\lstdefinestyle{matlab-style}{
language=Matlab,
basicstyle=\scriptsize\ttfamily,
tabsize=2,
rulecolor=,
language=matlab,
basicstyle=\scriptsize,
aboveskip={1.5\baselineskip},
columns=fullflexible,
showstringspaces=false,
extendedchars=true,
breaklines=true,
prebreak = \raisebox{0ex}[0ex][0ex]{\ensuremath{\hookleftarrow}},
frame=single,
showtabs=false,
showspaces=false,
showstringspaces=false,
identifierstyle=\ttfamily,
keywordstyle=\color[rgb]{0,0,1},
commentstyle=\color[rgb]{0.133,0.545,0.133},
stringstyle=\color[rgb]{0.627,0.126,0.941},
keepspaces=true,
numbers=left,
numbersep=5pt,
numberstyle=\tiny\color[rgb]{0.5,0.5,0.5},
stepnumber=1
}

\title{Design of an optimal heat exchanger\\MANE 6963 - Project 1}
\author{ID: 2168}
\date{}

\begin{document}

\maketitle

\section{Executive summary}

In this report, we investigate the optimal design of a heat exchanger that
heats air using hot water. An illustration of the geometry of the heat
exchanger is shown in Figure \ref{fig:exchanger}. The
goal of this investigation is to first define a geometry parameterization
$h = h(x, \boldsymbol{p})$ for the upper surface of the exchanger, where
$\boldsymbol{p} \in \mathbb{R}^{N_p}$ is the vector of design parameters
of length $N_p$. Then, using this geometry parameterization, we aim to
determine the parameter vector $\boldsymbol{p}$ that maximizes the heat
flux per unit length from the water to the air, $f(\boldsymbol{p})$.
\begin{figure}[htb]
\centering
\includegraphics[scale=0.5]{exchanger}
\caption{Geometry of the heat exchanger design problem}
\label{fig:exchanger}
\end{figure}

To achieve this goal, we have:
\begin{itemize}
\item Defined an appropriate geometry parameterization
$h = h(x, \boldsymbol{p})$ of the upper surface of the exchanger.
\item Implemented a Matlab routine \textsc{SurfaceHeight.m} that
computes the parameterized surface height $h$.
\item Utilized a finite volume analysis method to approximate
the objective function $f(\boldsymbol{p})$ at a given fixed
geometry parameterization $h$.
\item Utilized Matlab's \textsc{fmincon} routine to determine an optimal
set of design parameters $\boldsymbol{p}$.
\item Performed a mesh convergence study using the optimized geometry
to ensure that the flux $f(\boldsymbol{p})$ is accurately computed.
\end{itemize}

The remainder of this report presents details of the theory and
implementations of the items in the list above. A listing of the
implemented Matlab routines is provided in the appendix.

\section{Geometry parameterization}

To parameterize the geometry of the top surface of the heat exchanger,
we have chosen a linear combination of the squares of trigonometric functions
of the form:
\begin{equation}
h(x, \boldsymbol{p}) = p_1 + \sum_{k=0}^{N_p} p_k
\left[ \sin \left( \frac{2 \pi (k-1) x} {L} \right) \right]^2
\end{equation}
We note that this form of parameterization supports
at most $2(N_p - 1)$ modes in the upper surface geometry. We choose
the number of design variables to be $N_p = 5$ to support a
relatively complex geometry while still maintaining computational
feasibility for the optimization routines. The Matlab routine
\textsc{SurfaceHeight.m} computes the surface height at discrete
spatial locations given an input design parameter vector
$\boldsymbol{p}$.

\section{Analysis methods}

To determine the flux per unit area from water to air, we first
approximate the temperature everywhere in the domain by solving the steady
state heat flux problem: find $T$ such that:
\begin{equation}
\begin{cases}
\begin{aligned}
\nabla \cdot ( \kappa \nabla T) &= 0, \quad x \in \Omega, \\
T(x,y=0) &= T_{\text{water}}, \\
T(x,h(x,\boldsymbol{p})) &= T_{\text{air}}, \\
T(x,y) &= T(x + L, y)
\end{aligned}
\end{cases}
\end{equation}
with a finite volume method. Here, $T$ is the temperature in the heat
exchanger domain $\Omega$, $\kappa$ is the thermal conductivity of the
heat exchanger, $T_{\text{air}}$ is the temperature in the air, and
$T_{\text{water}}$ is the temperature in the water. Dirichlet boundary
conditions are applied to the top and bottom surfaces of the domain,
while periodic boundary conditions are applied to the left and right
boundaries.

We assume that the temperature in the air $T_{\text{air}}$,
the temperature in the water $T_{\text{water}}$,
and the thermal conductivity $\kappa$ do not vary spatially.
To accurately account for the geometrical complexity of the upper
surface,
we choose a uniform grid of $N_x = N_y = 15*2(N_p-1)$
finite volumes in the $x$ and $y$ directions, so that the finest mode
of the geometry is resolved by at least $15$ finite volumes.

Once the temperature is computed everywhere in the domain,
the flux
\begin{equation}
f(\boldsymbol{p}) = \kappa \int_0^L
\frac{\partial T(x,y=0,\boldsymbol{p})}{\partial y} \text{d} x
\end{equation}
from the air to the water is approximated
with a finite-difference approximation.

\section{Optimization problem statement}

Given the definitions of the surface height parameterization
$h(x, \boldsymbol{p})$ and the flux per unit area $f(\boldsymbol{p})$
from previous sections, the constrained optimization problem we wish
to solve is: 
\begin{equation}
\begin{aligned}
& \underset{\boldsymbol{p}}{\text{minimize}}
& & \left[ f(\boldsymbol{p}) \right]^{-1} \\
& \text{subject to}
& & h_{\text{min}} \leq h(x, \boldsymbol{p}) \leq h_{\text{max}}
\end{aligned}
\label{eq:optimize}
\end{equation}
Here, we have constrained the surface height to be greater than
or equal to a minimum value $h_{\text{min}}$ and less than or equal
to a maximum value $h_{\text{max}}$. These constraints are
implemented in the routine \textsc{NonlinearConstraints.m}
to be passed to Matlab's \textsc{fmincon} routine as nonlinear
constraints.

\section{Optimization methods}

We utilize Matlab's \textsc{fmincon} routine to solve the
optimization routine \eqref{eq:optimize}. This routine performs
gradient-based optimization, using a finite-difference approximation
of the gradient $\nabla f$. We provide \textsc{fmincon} with the
following options:
\begin{verbatim}
optns = optimoptions(@fmincon,...
     'Display','iter',...
     'Algorithm', 'sqp');
\end{verbatim}
Here we have chosen to use a sequentially quadratic programming
algorithm for the optimization routine, where we assume that the
objective function $f(\boldsymbol{p})$ is twice continuously
differentiable. The Matlab routine \textsc{MaxFluxDriver.m} solves
the optimization problem through a call to \textsc{fmincon}.

\section{Results}

We choose the length of the analysis domain $L=0.05$m, the
number of design variables $N_p = 5$, the temperature of the
air to be $T_\text{water} = 90 \degree$ C, the temperature of the
water to be $T_\text{air} = 20 \degree$ C, the thermal
conductivity to be $\kappa = 20 W / (\text{m} \degree \text{C})$,
the minimum surface height to be $h_{\text{min}} = 0.01$m, and
the maximum surface height to be $h_{\text{max}} = 0.05$m.

The routine \textsc{MaxFluxDriver.m} was called with the defined
inputs:
\begin{verbatim}
p = MaxFluxDriver(L,Np,Tair,Twater,kappa,hmin,hmax);
\end{verbatim}
and returned the output:
\begin{verbatim}
                                                          Norm of First-order
 Iter F-count            f(x) Feasibility  Steplength        step  optimality
    0       6    1.430717e-04   2.270e+00                           3.396e-04
    1      12    7.062792e-05   0.000e+00   1.000e+00   1.466e+00   9.445e-01
    2      18    6.806892e-05   4.410e-13   1.000e+00   1.586e-03   1.311e-03
    3      24    5.243068e-05   0.000e+00   1.000e+00   8.840e-03   1.733e-03
    4      30    5.142404e-05   0.000e+00   1.000e+00   1.777e-03   4.451e-04
    5      36    5.108063e-05   0.000e+00   1.000e+00   6.436e-04   7.174e-05
    6      42    5.107208e-05   0.000e+00   1.000e+00   1.050e-04   7.236e-05
    7      48    5.102902e-05   9.277e-14   1.000e+00   5.261e-04   7.548e-05
    8      54    5.083468e-05   1.423e-13   1.000e+00   2.323e-03   7.200e-05
    9      60    5.083412e-05   0.000e+00   1.000e+00   2.345e-05   1.847e-05
   10      66    5.083097e-05   0.000e+00   1.000e+00   1.211e-04   2.133e-05
   11      72    5.080318e-05   0.000e+00   1.000e+00   7.054e-04   3.803e-05
   12      78    5.061347e-05   1.571e-14   1.000e+00   2.284e-03   7.164e-05
   13      84    4.964888e-05   0.000e+00   1.000e+00   5.929e-03   1.563e-04
   14      90    4.811312e-05   0.000e+00   1.000e+00   6.278e-03   1.701e-04
   15      96    4.655550e-05   0.000e+00   1.000e+00   8.176e-03   2.697e-04
   16     102    4.079912e-05   0.000e+00   1.000e+00   1.433e-02   5.543e-04
   17     108    4.076109e-05   5.204e-18   1.000e+00   2.133e-04   2.596e-04
   18     114    4.057036e-05   1.735e-18   1.000e+00   1.067e-03   2.422e-04
   19     120    4.049092e-05   1.735e-18   1.000e+00   4.435e-04   2.213e-05
   20     127    4.049092e-05   1.388e-17   1.000e+00   1.203e-13   9.802e-10
\end{verbatim}
where the optimal parameter vector $\boldsymbol{p}$ was found to be:
\begin{verbatim}
p =

   0.010000000000000
  -0.000000000000000
   0.000000000000000
   0.000000000000000
   0.040441876010831
\end{verbatim}
The final geometry of the upper surface is shown in Figure
\ref{fig:geometry}.

\begin{figure}[hbt]
\centering
\includegraphics[width=0.5\textwidth]{geometry}
\caption{Final upper surface geometry}
\label{fig:geometry}
\end{figure}

\section{Conclusions}

\newpage

\section{Appendix: code listings}

\lstinputlisting[
  style=matlab-style,
  caption=SurfaceHeight.m]{SurfaceHeight.m}

\newpage

\lstinputlisting[ 
  style=matlab-style,
  caption=NonlinearConstraints.m]{NonlinearConstraints.m}

\newpage

\lstinputlisting[
  style=matlab-style,
  caption=MaxFluxDriver.m]{MaxFluxDriver.m}

\end{document}
